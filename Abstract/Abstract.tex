% don't display the page number
%\thispagestyle{empty}


Physicists have been very interested in transport phenomena in disordered systems like crystals with impurities. The feature of eigenstates is of great importance for understanding transport phenomena in such systems. Hence, a numerical approach using finite difference method is used to solve one dimensional Schr\"{o}dinger equations for ordered and disordered systems. Comparisons are made in order to reveal a fundamental difference, namely localization properties, between these two systems. We explore other factors like potential height, atomic spacing to see their effect on degree of localization in our modified random Kronig Penney model that we will introduce in this paper. Moreover, another model in which we consider a chain of atoms connected by strings obeying Hooke's law is also explored with some randomness that we introduce. In their normal modes, we again find localization properties for these systems. To some extent, this indicates localization is a universal property for disordered systems. Finally, with the aid of finite difference method and Bloch theorem, we explore band structure for periodic Kronig Penney models with negative finite potential wells. In particular, we look at band gaps of these models and explore its relation with the periodic potential. 